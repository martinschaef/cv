\section{\sc Research Statement}


My research comprises different aspects of software quality. In the past years
my research has been focused in the areas of \emph{static error detection},
\emph{fault localization}, and \emph{system testing}:

\noindent\emph{Static Error Detection.} Driven by the wish for verified
software, static program analysis has been an active field of research over the
past decades. Despite major achievements, usability remains an open problem.
Existing analysis tools either require a tremendous amount of user interaction
and special training, making them only applicable to extremely safety-critical
software (e.g., the Microsoft Static Driver Verifier), or they produce a
prohibitively large number of false alarms.


Over the past few years, I have developed a static analysis tool called Joogie
(\url{www.joogie.org}). Joogie identifies common coding mistakes such as
unreachable code, use of non-allocated memory, or contradicting or useless 
control-flow. To this end, Joogie tries, for each statement in a program, to
compute a proof that no execution passing this statement can terminate
un-exceptionally.

Such a proof can be computed locally as it holds for \emph{any} execution of
a statement, and thus still holds in a larger context. Hence, the analysis
can be applied in an extremely modular way, and runs with a close-to-zero false
alarms.

Joogie is still in active development and has already been applied to
real-world applications. It is able to find bugs and unreachable code that has been
previously undetected.

\noindent\emph{Fault Localization.} Once an error is found, chasing down the
statements that need to be changed to eliminate the error is a daunting task. I
have developed a novel approach  which, for a given error trace, computes
invariant properties that hold along parts of the trace. It provides a guarantee
that any statement for which this invariant property holds does not need to be
considered to fix the observed error. This allows a compact representation of
the error trace with additional information provided by the invariant
properties. The approach is promising, and recent experiments indicate a
significant speed up during debugging. An integration in Joogie is in the making.

\noindent\emph{Grey-box GUI testing.} While our research shows that static
analysis is suitable for code snippets, it fails to analyze large systems due
to the complexity of the analysis. Reactive systems such as systems with graphical
user interfaces are an interesting challenge here, due to their possibly infinite
number of user interactions.   

I am participating in the development of GUI testing techniques that
utilize information obtained from light weight static analysis to generate
\emph{suitable} interaction sequences and test data. The combination of
light-weight analysis and system level testing has proved useful for larger
application, where static analysis alone cannot cope with the complexity, and
black-box testing fails due to the extreme number of test cases.

